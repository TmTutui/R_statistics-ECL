% Options for packages loaded elsewhere
\PassOptionsToPackage{unicode}{hyperref}
\PassOptionsToPackage{hyphens}{url}
%
\documentclass[
]{article}
\usepackage{amsmath,amssymb}
\usepackage{lmodern}
\usepackage{iftex}
\ifPDFTeX
  \usepackage[T1]{fontenc}
  \usepackage[utf8]{inputenc}
  \usepackage{textcomp} % provide euro and other symbols
\else % if luatex or xetex
  \usepackage{unicode-math}
  \defaultfontfeatures{Scale=MatchLowercase}
  \defaultfontfeatures[\rmfamily]{Ligatures=TeX,Scale=1}
\fi
% Use upquote if available, for straight quotes in verbatim environments
\IfFileExists{upquote.sty}{\usepackage{upquote}}{}
\IfFileExists{microtype.sty}{% use microtype if available
  \usepackage[]{microtype}
  \UseMicrotypeSet[protrusion]{basicmath} % disable protrusion for tt fonts
}{}
\makeatletter
\@ifundefined{KOMAClassName}{% if non-KOMA class
  \IfFileExists{parskip.sty}{%
    \usepackage{parskip}
  }{% else
    \setlength{\parindent}{0pt}
    \setlength{\parskip}{6pt plus 2pt minus 1pt}}
}{% if KOMA class
  \KOMAoptions{parskip=half}}
\makeatother
\usepackage{xcolor}
\usepackage[margin=1in]{geometry}
\usepackage{color}
\usepackage{fancyvrb}
\newcommand{\VerbBar}{|}
\newcommand{\VERB}{\Verb[commandchars=\\\{\}]}
\DefineVerbatimEnvironment{Highlighting}{Verbatim}{commandchars=\\\{\}}
% Add ',fontsize=\small' for more characters per line
\usepackage{framed}
\definecolor{shadecolor}{RGB}{248,248,248}
\newenvironment{Shaded}{\begin{snugshade}}{\end{snugshade}}
\newcommand{\AlertTok}[1]{\textcolor[rgb]{0.94,0.16,0.16}{#1}}
\newcommand{\AnnotationTok}[1]{\textcolor[rgb]{0.56,0.35,0.01}{\textbf{\textit{#1}}}}
\newcommand{\AttributeTok}[1]{\textcolor[rgb]{0.77,0.63,0.00}{#1}}
\newcommand{\BaseNTok}[1]{\textcolor[rgb]{0.00,0.00,0.81}{#1}}
\newcommand{\BuiltInTok}[1]{#1}
\newcommand{\CharTok}[1]{\textcolor[rgb]{0.31,0.60,0.02}{#1}}
\newcommand{\CommentTok}[1]{\textcolor[rgb]{0.56,0.35,0.01}{\textit{#1}}}
\newcommand{\CommentVarTok}[1]{\textcolor[rgb]{0.56,0.35,0.01}{\textbf{\textit{#1}}}}
\newcommand{\ConstantTok}[1]{\textcolor[rgb]{0.00,0.00,0.00}{#1}}
\newcommand{\ControlFlowTok}[1]{\textcolor[rgb]{0.13,0.29,0.53}{\textbf{#1}}}
\newcommand{\DataTypeTok}[1]{\textcolor[rgb]{0.13,0.29,0.53}{#1}}
\newcommand{\DecValTok}[1]{\textcolor[rgb]{0.00,0.00,0.81}{#1}}
\newcommand{\DocumentationTok}[1]{\textcolor[rgb]{0.56,0.35,0.01}{\textbf{\textit{#1}}}}
\newcommand{\ErrorTok}[1]{\textcolor[rgb]{0.64,0.00,0.00}{\textbf{#1}}}
\newcommand{\ExtensionTok}[1]{#1}
\newcommand{\FloatTok}[1]{\textcolor[rgb]{0.00,0.00,0.81}{#1}}
\newcommand{\FunctionTok}[1]{\textcolor[rgb]{0.00,0.00,0.00}{#1}}
\newcommand{\ImportTok}[1]{#1}
\newcommand{\InformationTok}[1]{\textcolor[rgb]{0.56,0.35,0.01}{\textbf{\textit{#1}}}}
\newcommand{\KeywordTok}[1]{\textcolor[rgb]{0.13,0.29,0.53}{\textbf{#1}}}
\newcommand{\NormalTok}[1]{#1}
\newcommand{\OperatorTok}[1]{\textcolor[rgb]{0.81,0.36,0.00}{\textbf{#1}}}
\newcommand{\OtherTok}[1]{\textcolor[rgb]{0.56,0.35,0.01}{#1}}
\newcommand{\PreprocessorTok}[1]{\textcolor[rgb]{0.56,0.35,0.01}{\textit{#1}}}
\newcommand{\RegionMarkerTok}[1]{#1}
\newcommand{\SpecialCharTok}[1]{\textcolor[rgb]{0.00,0.00,0.00}{#1}}
\newcommand{\SpecialStringTok}[1]{\textcolor[rgb]{0.31,0.60,0.02}{#1}}
\newcommand{\StringTok}[1]{\textcolor[rgb]{0.31,0.60,0.02}{#1}}
\newcommand{\VariableTok}[1]{\textcolor[rgb]{0.00,0.00,0.00}{#1}}
\newcommand{\VerbatimStringTok}[1]{\textcolor[rgb]{0.31,0.60,0.02}{#1}}
\newcommand{\WarningTok}[1]{\textcolor[rgb]{0.56,0.35,0.01}{\textbf{\textit{#1}}}}
\usepackage{graphicx}
\makeatletter
\def\maxwidth{\ifdim\Gin@nat@width>\linewidth\linewidth\else\Gin@nat@width\fi}
\def\maxheight{\ifdim\Gin@nat@height>\textheight\textheight\else\Gin@nat@height\fi}
\makeatother
% Scale images if necessary, so that they will not overflow the page
% margins by default, and it is still possible to overwrite the defaults
% using explicit options in \includegraphics[width, height, ...]{}
\setkeys{Gin}{width=\maxwidth,height=\maxheight,keepaspectratio}
% Set default figure placement to htbp
\makeatletter
\def\fps@figure{htbp}
\makeatother
\setlength{\emergencystretch}{3em} % prevent overfull lines
\providecommand{\tightlist}{%
  \setlength{\itemsep}{0pt}\setlength{\parskip}{0pt}}
\setcounter{secnumdepth}{-\maxdimen} % remove section numbering
\ifLuaTeX
  \usepackage{selnolig}  % disable illegal ligatures
\fi
\IfFileExists{bookmark.sty}{\usepackage{bookmark}}{\usepackage{hyperref}}
\IfFileExists{xurl.sty}{\usepackage{xurl}}{} % add URL line breaks if available
\urlstyle{same} % disable monospaced font for URLs
\hypersetup{
  pdftitle={BE3 - partie 1},
  pdfauthor={Tulio NAVARRO TUTUI, Filipe PENNA CERAVOLO SOARES},
  hidelinks,
  pdfcreator={LaTeX via pandoc}}

\title{BE3 - partie 1}
\author{Tulio NAVARRO TUTUI, Filipe PENNA CERAVOLO SOARES}
\date{06 December, 2022}

\begin{document}
\maketitle

On reprend le jeu de données concernant la valeur des logements des
villes aux alentours de Boston. On cherche à identifier un bon modèle
part régression pénalisée, CART, Boosting et Random Forest.

Les variables utilisées sont les suivantes:

\textbf{CRIM} taux de criminalité par habitant

\textbf{ZN} proportion de terrains résidentiels

\textbf{INDUS} proportion de terrains industriels

\textbf{CHAS} 1 si ville en bordure de la rivière Charles 0 sinon

\textbf{NOX} concentration en oxydes d'azote

\textbf{RM} nombre moyen de pièces par logement

\textbf{AGE} proportion de logements construit avant 1940

\textbf{DIS} distance du centre de Boston

\textbf{RAD} accessibilité aux autoroutes de contournement

\textbf{TAX} taux de l'impôt foncier

\textbf{PTRATIO} rapport élèves-enseignant par ville

\textbf{LSTAT} \% de la population `a faibles revenus

\textbf{\(class\)} valeur du logement en 1000\$

L'objectif de ce BE est de comparer différents modèles de machines
learning : le modèle linéaire, le modèle linéaire pénalisé (Ridge et
Lasso), les modèles linéaires de réduction de dimension (PCR et PLS),
les modèles à base d'arbres (CART, Boosting, Bagging et Random Forest).

\hypertarget{moduxe8les-linuxe9aires}{%
\section{Modèles linéaires}\label{moduxe8les-linuxe9aires}}

On commence par ajuster un modèle linéaire sans et avec interactions à
partir de l'ensemble d'apprentissage constitué de 300 observations.
Evaluer la qualité prédictive de ces deux modèles sur l'ensemble test
constitué de 206 observations. On pourra utiliser la commande RMSE du
package DiceEval.

\begin{Shaded}
\begin{Highlighting}[]
\FunctionTok{library}\NormalTok{(DiceEval)}
\end{Highlighting}
\end{Shaded}

\begin{verbatim}
## Loading required package: DiceKriging
\end{verbatim}

\begin{Shaded}
\begin{Highlighting}[]
\FunctionTok{library}\NormalTok{(car)}
\end{Highlighting}
\end{Shaded}

\begin{verbatim}
## Loading required package: carData
\end{verbatim}

\begin{Shaded}
\begin{Highlighting}[]
\FunctionTok{library}\NormalTok{(MASS)}

\NormalTok{housing }\OtherTok{=} \FunctionTok{read.table}\NormalTok{(}\StringTok{"housing\_new.txt"}\NormalTok{, }\AttributeTok{header =}\NormalTok{ T)}
\NormalTok{p }\OtherTok{=} \FunctionTok{ncol}\NormalTok{(housing)}
\FunctionTok{summary}\NormalTok{(housing)}
\end{Highlighting}
\end{Shaded}

\begin{verbatim}
##       CRIM               ZN             INDUS            CHAS        
##  Min.   : 0.0060   Min.   :  0.00   Min.   : 0.46   Min.   :0.00000  
##  1st Qu.: 0.0820   1st Qu.:  0.00   1st Qu.: 5.19   1st Qu.:0.00000  
##  Median : 0.2565   Median :  0.00   Median : 9.69   Median :0.00000  
##  Mean   : 3.6135   Mean   : 11.36   Mean   :11.14   Mean   :0.06917  
##  3rd Qu.: 3.6770   3rd Qu.: 12.50   3rd Qu.:18.10   3rd Qu.:0.00000  
##  Max.   :88.9760   Max.   :100.00   Max.   :27.74   Max.   :1.00000  
##       NOX               RM             AGE              DIS        
##  Min.   :0.3850   Min.   :3.561   Min.   :  2.90   Min.   : 1.130  
##  1st Qu.:0.4490   1st Qu.:5.886   1st Qu.: 45.02   1st Qu.: 2.100  
##  Median :0.5380   Median :6.208   Median : 77.50   Median : 3.208  
##  Mean   :0.5547   Mean   :6.285   Mean   : 68.57   Mean   : 3.795  
##  3rd Qu.:0.6240   3rd Qu.:6.623   3rd Qu.: 94.08   3rd Qu.: 5.189  
##  Max.   :0.8710   Max.   :8.780   Max.   :100.00   Max.   :12.126  
##       RAD              TAX           PTRATIO          LSTAT      
##  Min.   : 1.000   Min.   :187.0   Min.   :12.60   Min.   : 1.73  
##  1st Qu.: 4.000   1st Qu.:279.0   1st Qu.:17.40   1st Qu.: 6.95  
##  Median : 5.000   Median :330.0   Median :19.05   Median :11.36  
##  Mean   : 9.549   Mean   :408.2   Mean   :18.46   Mean   :12.65  
##  3rd Qu.:24.000   3rd Qu.:666.0   3rd Qu.:20.20   3rd Qu.:16.95  
##  Max.   :24.000   Max.   :711.0   Max.   :22.00   Max.   :37.97  
##      class      
##  Min.   : 5.00  
##  1st Qu.:17.02  
##  Median :21.20  
##  Mean   :22.53  
##  3rd Qu.:25.00  
##  Max.   :50.00
\end{verbatim}

\begin{Shaded}
\begin{Highlighting}[]
\FunctionTok{dim}\NormalTok{(housing)}
\end{Highlighting}
\end{Shaded}

\begin{verbatim}
## [1] 506  13
\end{verbatim}

\begin{Shaded}
\begin{Highlighting}[]
\FunctionTok{set.seed}\NormalTok{(}\DecValTok{23}\NormalTok{)}
\NormalTok{u }\OtherTok{=} \FunctionTok{sample}\NormalTok{(}\DecValTok{1}\SpecialCharTok{:}\FunctionTok{nrow}\NormalTok{(housing), }\DecValTok{300}\NormalTok{)}
\NormalTok{housing.train }\OtherTok{=}\NormalTok{ housing[u,]}
\NormalTok{housing.test }\OtherTok{=}\NormalTok{ housing[}\SpecialCharTok{{-}}\NormalTok{u,]}
\end{Highlighting}
\end{Shaded}

\begin{Shaded}
\begin{Highlighting}[]
\FunctionTok{library}\NormalTok{(caret)}
\end{Highlighting}
\end{Shaded}

\begin{verbatim}
## Loading required package: ggplot2
\end{verbatim}

\begin{verbatim}
## Loading required package: lattice
\end{verbatim}

\begin{verbatim}
## 
## Attaching package: 'caret'
\end{verbatim}

\begin{verbatim}
## The following objects are masked from 'package:DiceEval':
## 
##     MAE, R2, RMSE
\end{verbatim}

\begin{Shaded}
\begin{Highlighting}[]
\CommentTok{\# Ajuster un modèle linéaire simple sans interactions}
\NormalTok{mod\_lm }\OtherTok{\textless{}{-}} \FunctionTok{lm}\NormalTok{(class }\SpecialCharTok{\textasciitilde{}}\NormalTok{., }\AttributeTok{data =}\NormalTok{ housing.train)}
\NormalTok{Y\_pred\_lm }\OtherTok{=} \FunctionTok{predict}\NormalTok{(mod\_lm, housing.test[, }\SpecialCharTok{{-}}\NormalTok{p])}
\NormalTok{RMSE\_lm }\OtherTok{=} \FunctionTok{RMSE}\NormalTok{(housing.test[, p], Y\_pred\_lm)  }\CommentTok{\# root mean square error}
\NormalTok{RMSE\_lm }\CommentTok{\# 5.9}
\end{Highlighting}
\end{Shaded}

\begin{verbatim}
## [1] 5.900007
\end{verbatim}

\begin{Shaded}
\begin{Highlighting}[]
\CommentTok{\# Ajuster un modèle linéaire avec interactions}
\CommentTok{\#a completer :}
\NormalTok{mod\_lm\_inter }\OtherTok{\textless{}{-}} \FunctionTok{lm}\NormalTok{(class }\SpecialCharTok{\textasciitilde{}}\NormalTok{.}\SpecialCharTok{\^{}}\DecValTok{2}\NormalTok{, }\AttributeTok{data =}\NormalTok{ housing.train)}
\FunctionTok{length}\NormalTok{(}\FunctionTok{na.omit}\NormalTok{(mod\_lm\_inter}\SpecialCharTok{$}\NormalTok{coefficients))}
\end{Highlighting}
\end{Shaded}

\begin{verbatim}
## [1] 79
\end{verbatim}

\begin{Shaded}
\begin{Highlighting}[]
\CommentTok{\#a completer :}
\NormalTok{Y\_pred\_lm\_inter }\OtherTok{=} \FunctionTok{predict}\NormalTok{(mod\_lm\_inter, housing.test[, }\SpecialCharTok{{-}}\NormalTok{p])}
\CommentTok{\#a completer :}
\NormalTok{RMSE\_lm\_inter }\OtherTok{=} \FunctionTok{RMSE}\NormalTok{(housing.test[, p], Y\_pred\_lm\_inter)}
\NormalTok{RMSE\_lm\_inter  }\CommentTok{\# 4.41}
\end{Highlighting}
\end{Shaded}

\begin{verbatim}
## [1] 4.416989
\end{verbatim}

On remarque qu'en ajoutant les termes d'interaction l'erreur a diminué.
Cependant le modèle avec interaction est très complexe (79 termes) donc
certainement trop variable. L'idée est de chercher un modèle plus
prédictif, i.e.~avec une RMSE plus faible que 4.836.

\hypertarget{moduxe8les-linuxe9aires-puxe9nalisuxe9s}{%
\section{Modèles linéaires
pénalisés}\label{moduxe8les-linuxe9aires-puxe9nalisuxe9s}}

Ajuster une régression lasso (commande lars du package lars ). Pour ce
faire, il est nécessaire de créer une matrice contenant tous les termes
(variables principales et les interactions).

\begin{Shaded}
\begin{Highlighting}[]
\CommentTok{\#{-}{-}{-}{-}{-}{-}{-}{-}{-}{-}{-}{-}{-}{-}{-}{-}{-}{-}{-}{-}{-}{-}{-}{-}{-}{-}{-}{-}{-}{-}{-}{-}{-}{-}{-}{-}{-}{-}{-}{-}{-}{-}{-}{-}{-}{-}{-}{-}{-}{-}{-}{-}{-}{-}{-}{-}{-}{-}{-}{-}{-}{-}}
\CommentTok{\# lasso}
\CommentTok{\#{-}{-}{-}{-}{-}{-}{-}{-}{-}{-}{-}{-}{-}{-}{-}{-}{-}{-}{-}{-}{-}{-}{-}{-}{-}{-}{-}{-}{-}{-}{-}{-}{-}{-}{-}{-}{-}{-}{-}{-}{-}{-}{-}{-}{-}{-}{-}{-}{-}{-}{-}{-}{-}{-}{-}{-}{-}{-}{-}{-}{-}{-}}

\FunctionTok{library}\NormalTok{(lars)}
\end{Highlighting}
\end{Shaded}

\begin{verbatim}
## Loaded lars 1.3
\end{verbatim}

\begin{Shaded}
\begin{Highlighting}[]
\NormalTok{y }\OtherTok{=} \FunctionTok{as.matrix}\NormalTok{(housing.train[, p])}
\NormalTok{x }\OtherTok{=} \FunctionTok{as.matrix}\NormalTok{(housing.train[, }\SpecialCharTok{{-}}\NormalTok{p])}
\NormalTok{x\_extend }\OtherTok{=}\NormalTok{ x}

\CommentTok{\# makes the binary combinations of all columns}
\ControlFlowTok{for}\NormalTok{ (i }\ControlFlowTok{in} \DecValTok{1}\SpecialCharTok{:}\NormalTok{(p}\DecValTok{{-}2}\NormalTok{)) \{}
  \ControlFlowTok{for}\NormalTok{ (j }\ControlFlowTok{in} \DecValTok{2}\SpecialCharTok{:}\NormalTok{(p}\DecValTok{{-}1}\NormalTok{)) \{}
\NormalTok{    x\_extend }\OtherTok{=} \FunctionTok{cbind}\NormalTok{(x\_extend, x[, i] }\SpecialCharTok{*}\NormalTok{ x[, j])}
\NormalTok{  \}}
\NormalTok{\}}

\NormalTok{mod\_lasso }\OtherTok{=} \FunctionTok{lars}\NormalTok{(x\_extend, y, }\AttributeTok{type =} \StringTok{"lasso"}\NormalTok{)}
\end{Highlighting}
\end{Shaded}

--- Etudier la matrice des coefficients (commande coef.lars) et tracer
le modèle (commande plot).

\begin{Shaded}
\begin{Highlighting}[]
\FunctionTok{dim}\NormalTok{(}\FunctionTok{coef.lars}\NormalTok{(mod\_lasso)) }\CommentTok{\# 186 valeurs de lambda différentes et 133 coefficients}
\end{Highlighting}
\end{Shaded}

\begin{verbatim}
## [1] 186 133
\end{verbatim}

\begin{Shaded}
\begin{Highlighting}[]
\CommentTok{\# Faire un choix de ligne entre 1 (très forte pénalité) et 186 (faible pénalité)}
\NormalTok{l }\OtherTok{=} \DecValTok{10}
\FunctionTok{sum}\NormalTok{(}\FunctionTok{coef.lars}\NormalTok{(mod\_lasso)[l, ] }\SpecialCharTok{==} \DecValTok{0}\NormalTok{)  }\CommentTok{\# n de coefs = 0}
\end{Highlighting}
\end{Shaded}

\begin{verbatim}
## [1] 127
\end{verbatim}

With l = 10, sum equals 127, whareas with l = 100, sum equals 82. This
shows that with a lesser l, more coefficients are getting 0 as a value.

\begin{Shaded}
\begin{Highlighting}[]
\FunctionTok{par}\NormalTok{(}\AttributeTok{mfrow =} \FunctionTok{c}\NormalTok{(}\DecValTok{1}\NormalTok{, }\DecValTok{1}\NormalTok{))}
\FunctionTok{plot}\NormalTok{(mod\_lasso, }\AttributeTok{cex =} \DecValTok{1}\NormalTok{, }\AttributeTok{lwd =} \DecValTok{3}\NormalTok{, }\AttributeTok{col =} \DecValTok{1}\SpecialCharTok{:}\NormalTok{(p}\DecValTok{{-}1}\NormalTok{))}
\FunctionTok{legend}\NormalTok{(}
  \DecValTok{0}\NormalTok{,}
  \DecValTok{0}\NormalTok{,}
  \FunctionTok{names}\NormalTok{(housing.train[, }\SpecialCharTok{{-}}\NormalTok{p]),}
  \AttributeTok{col =} \DecValTok{1}\SpecialCharTok{:}\NormalTok{(p}\DecValTok{{-}1}\NormalTok{),}
  \AttributeTok{lty =} \FunctionTok{rep}\NormalTok{(}\DecValTok{1}\NormalTok{, p}\DecValTok{{-}1}\NormalTok{),}
  \AttributeTok{lwd =} \FunctionTok{rep}\NormalTok{(}\DecValTok{3}\NormalTok{, p}\DecValTok{{-}1}\NormalTok{),}
  \AttributeTok{cex =} \FloatTok{0.5}
\NormalTok{)}
\end{Highlighting}
\end{Shaded}

\includegraphics{BE3_partie1_incomplet_files/figure-latex/unnamed-chunk-5-1.pdf}

--- Faire un choix du meilleur modèle en étudiant la décroissance de
l'erreur par validation croisée (commande cv.lars). Faire un choix de la
meilleure fraction (directement liée à la valeur du lambda. Fraction = 1
(resp. Fraction = 0) correspond à \(\lambda = 0\) (resp.
\(\lambda = +\infty\))).

\begin{Shaded}
\begin{Highlighting}[]
\NormalTok{CV }\OtherTok{=} \FunctionTok{cv.lars}\NormalTok{(x, y, }\AttributeTok{K =} \DecValTok{10}\NormalTok{, }\AttributeTok{index=}\FunctionTok{seq}\NormalTok{(}\AttributeTok{from =} \DecValTok{0}\NormalTok{, }\AttributeTok{to =} \DecValTok{1}\NormalTok{, }\AttributeTok{length =} \DecValTok{100}\NormalTok{),}
             \AttributeTok{trace =} \ConstantTok{FALSE}\NormalTok{, }\AttributeTok{plot.it =} \ConstantTok{TRUE}\NormalTok{, }\AttributeTok{se =} \ConstantTok{TRUE}\NormalTok{, }\AttributeTok{type =} \StringTok{"lasso"}\NormalTok{, }\AttributeTok{mode =} \StringTok{"fraction"}\NormalTok{)}
\end{Highlighting}
\end{Shaded}

\includegraphics{BE3_partie1_incomplet_files/figure-latex/unnamed-chunk-6-1.pdf}

\begin{Shaded}
\begin{Highlighting}[]
\CommentTok{\#a completer :}
\NormalTok{value }\OtherTok{=} \FloatTok{0.4}
\end{Highlighting}
\end{Shaded}

\begin{itemize}
\tightlist
\item
  Donner le nombre de coefficients non nuls du modèle final retenu. La
  pénalité a-t-elle jouer son rôle ?
\end{itemize}

\begin{Shaded}
\begin{Highlighting}[]
\CommentTok{\#a completer :}
\NormalTok{coef\_lasso\_beter }\OtherTok{\textless{}{-}} \FunctionTok{predict.lars}\NormalTok{(mod\_lasso, }\AttributeTok{s =}\NormalTok{ value, }\AttributeTok{mode =} \StringTok{"frac"}\NormalTok{, }\AttributeTok{type =} \StringTok{"coef"}\NormalTok{)}
\CommentTok{\#a completer :}
\NormalTok{nb\_coef\_nul }\OtherTok{=} \FunctionTok{sum}\NormalTok{(coef\_lasso\_beter}\SpecialCharTok{$}\NormalTok{coefficients }\SpecialCharTok{==} \DecValTok{0}\NormalTok{) }
\NormalTok{nb\_coef\_nul}
\end{Highlighting}
\end{Shaded}

\begin{verbatim}
## [1] 51
\end{verbatim}

Obtemos 51 coeficientes nulos. Um valor menor que os valores que
obtivemos antes. L portanto seria maior que 100.

--- Evaluer la qualité prédictive du modèle sur l'ensemble test
(commande predict.lars ).

\begin{Shaded}
\begin{Highlighting}[]
\NormalTok{newx }\OtherTok{=} \FunctionTok{as.matrix}\NormalTok{(housing.test[,}\SpecialCharTok{{-}}\NormalTok{p])}
\ControlFlowTok{for}\NormalTok{ (i }\ControlFlowTok{in} \DecValTok{1}\SpecialCharTok{:}\NormalTok{(p}\DecValTok{{-}2}\NormalTok{))}
\NormalTok{\{}
  \ControlFlowTok{for}\NormalTok{ (j }\ControlFlowTok{in} \DecValTok{2}\SpecialCharTok{:}\NormalTok{(p}\DecValTok{{-}1}\NormalTok{))}
\NormalTok{  \{}
\NormalTok{    newx }\OtherTok{=} \FunctionTok{cbind}\NormalTok{(newx,newx[,i]}\SpecialCharTok{*}\NormalTok{newx[,j])}
\NormalTok{  \}}
\NormalTok{\}}
\CommentTok{\#a completer :}
\NormalTok{fits }\OtherTok{\textless{}{-}} \FunctionTok{predict.lars}\NormalTok{(mod\_lasso, newx, }\AttributeTok{s =}\NormalTok{ value, }\AttributeTok{mode =} \StringTok{"frac"}\NormalTok{)}
\CommentTok{\#a completer :}
\NormalTok{RMSE\_lasso }\OtherTok{=} \FunctionTok{RMSE}\NormalTok{(housing.test[, p], fits}\SpecialCharTok{$}\NormalTok{fit)}
\FunctionTok{c}\NormalTok{(RMSE\_lm\_inter, RMSE\_lasso)}
\end{Highlighting}
\end{Shaded}

\begin{verbatim}
## [1] 4.416989 4.581404
\end{verbatim}

Ajuster maintenant une régression Ridge. La pénalité Ridge ne s'applique
pas sur l'intercept. Il est donc nécessaire d'enlever le terme constant
avant d'appliquer la procédure lm.ridge de la library MASS.

\begin{Shaded}
\begin{Highlighting}[]
\CommentTok{\#{-}{-}{-}{-}{-}{-}{-}{-}{-}{-}{-}{-}{-}{-}{-}{-}{-}{-}{-}{-}{-}{-}{-}{-}{-}{-}{-}{-}{-}{-}{-}{-}{-}{-}{-}{-}{-}{-}{-}{-}{-}{-}}
\CommentTok{\#ajustement ridge }
\CommentTok{\#{-}{-}{-}{-}{-}{-}{-}{-}{-}{-}{-}{-}{-}{-}{-}{-}{-}{-}{-}{-}{-}{-}{-}{-}{-}{-}{-}{-}{-}{-}{-}{-}{-}{-}{-}{-}{-}{-}{-}{-}{-}{-}}
\FunctionTok{library}\NormalTok{(MASS)}
\CommentTok{\#on part d\textquotesingle{}une matrice centrée réduite}
\NormalTok{housing.train2 }\OtherTok{=} \FunctionTok{as.data.frame}\NormalTok{(}\FunctionTok{scale}\NormalTok{(}\FunctionTok{cbind}\NormalTok{(x\_extend, y))) }\CommentTok{\# scale does the normalization of columns}
\CommentTok{\#on garde en mémoire moyenne et écart type des données initiales}
\NormalTok{mean.train }\OtherTok{=} \FunctionTok{apply}\NormalTok{(}\FunctionTok{cbind}\NormalTok{(x\_extend, y), }\DecValTok{2}\NormalTok{, mean)  }\CommentTok{\# 2 = mean is by column}
\NormalTok{sd.train }\OtherTok{=} \FunctionTok{apply}\NormalTok{(}\FunctionTok{cbind}\NormalTok{(x\_extend, y), }\DecValTok{2}\NormalTok{, sd)  }\CommentTok{\# 2 = sd is by column}

\CommentTok{\#On se donne un vecteur de poids lambda}
\NormalTok{lambda }\OtherTok{=} \FunctionTok{c}\NormalTok{(}\FunctionTok{seq}\NormalTok{(}\FloatTok{0.01}\NormalTok{, }\DecValTok{10}\NormalTok{, }\FloatTok{0.01}\NormalTok{), }\FunctionTok{seq}\NormalTok{(}\DecValTok{10}\NormalTok{, }\DecValTok{100}\NormalTok{, }\DecValTok{10}\NormalTok{), }\DecValTok{1000}\NormalTok{, }\DecValTok{2000}\NormalTok{)}

\NormalTok{p2 }\OtherTok{=} \FunctionTok{ncol}\NormalTok{(housing.train2)}
\NormalTok{mod\_ridge }\OtherTok{=} \FunctionTok{lm.ridge}\NormalTok{(V134 }\SpecialCharTok{\textasciitilde{}}\NormalTok{.}\SpecialCharTok{{-}}\DecValTok{1}\NormalTok{, }\AttributeTok{data =}\NormalTok{ housing.train2, }\AttributeTok{lambda =}\NormalTok{ lambda) }
\CommentTok{\# V134 is the class column with centre reduit}
\end{Highlighting}
\end{Shaded}

\begin{itemize}
\tightlist
\item
  Tracer les différents beta en fonction du \(\lambda\). Que
  remarquez-vous ?
\end{itemize}

\begin{Shaded}
\begin{Highlighting}[]
\CommentTok{\#tracer des différentes valeurs des beta en fonction de lambda}
\NormalTok{beta\_mat }\OtherTok{=}\NormalTok{ mod\_ridge}\SpecialCharTok{$}\NormalTok{coef}
\FunctionTok{par}\NormalTok{(}\AttributeTok{mfrow =} \FunctionTok{c}\NormalTok{(}\DecValTok{1}\NormalTok{, }\DecValTok{1}\NormalTok{))}
\FunctionTok{plot}\NormalTok{(lambda, beta\_mat[}\DecValTok{1}\NormalTok{, ], }\AttributeTok{ylim =} \FunctionTok{c}\NormalTok{(}\FunctionTok{min}\NormalTok{(beta\_mat),}
     \FunctionTok{max}\NormalTok{(beta\_mat)), }\AttributeTok{pch =} \DecValTok{19}\NormalTok{, }\AttributeTok{cex =} \FloatTok{0.4}\NormalTok{, }\AttributeTok{ylab =} \StringTok{"coefficients"}\NormalTok{, }\AttributeTok{res =} \DecValTok{600}
\NormalTok{)}
\end{Highlighting}
\end{Shaded}

\begin{verbatim}
## Warning in plot.window(...): "res" is not a graphical parameter
\end{verbatim}

\begin{verbatim}
## Warning in plot.xy(xy, type, ...): "res" is not a graphical parameter
\end{verbatim}

\begin{verbatim}
## Warning in axis(side = side, at = at, labels = labels, ...): "res" is not a
## graphical parameter

## Warning in axis(side = side, at = at, labels = labels, ...): "res" is not a
## graphical parameter
\end{verbatim}

\begin{verbatim}
## Warning in box(...): "res" is not a graphical parameter
\end{verbatim}

\begin{verbatim}
## Warning in title(...): "res" is not a graphical parameter
\end{verbatim}

\begin{Shaded}
\begin{Highlighting}[]
\FunctionTok{lines}\NormalTok{(lambda, beta\_mat[}\DecValTok{1}\NormalTok{, ], }\AttributeTok{lwd =} \DecValTok{2}\NormalTok{)}

\ControlFlowTok{for}\NormalTok{ (i }\ControlFlowTok{in} \DecValTok{2}\SpecialCharTok{:}\NormalTok{(p}\DecValTok{{-}1}\NormalTok{))}
\NormalTok{\{}
  \FunctionTok{points}\NormalTok{(lambda, beta\_mat[i, ], }\AttributeTok{col =}\NormalTok{ i, }\AttributeTok{pch =} \DecValTok{19}\NormalTok{, }\AttributeTok{cex=}\FloatTok{0.4}\NormalTok{)}
  \FunctionTok{lines}\NormalTok{(lambda, beta\_mat[i, ], }\AttributeTok{col =}\NormalTok{ i, }\AttributeTok{lwd =} \DecValTok{2}\NormalTok{)}
  
\NormalTok{\}}

\FunctionTok{legend}\NormalTok{(}\DecValTok{1000}\NormalTok{, }\FloatTok{2.5}\NormalTok{, }\FunctionTok{row.names}\NormalTok{(beta\_mat),}
       \AttributeTok{col =} \DecValTok{1}\SpecialCharTok{:}\NormalTok{p, }\AttributeTok{lty =} \FunctionTok{rep}\NormalTok{(}\DecValTok{1}\NormalTok{, p), }\AttributeTok{cex =} \FloatTok{0.5}\NormalTok{, }\AttributeTok{pch =}  \FunctionTok{rep}\NormalTok{(}\DecValTok{19}\NormalTok{, }\DecValTok{8}\NormalTok{)}
\NormalTok{)}
\end{Highlighting}
\end{Shaded}

\includegraphics{BE3_partie1_incomplet_files/figure-latex/unnamed-chunk-10-1.pdf}

Nesse caso os termos não vão diretamente à zero, por mais que se
aproximem dele.

Le choix du meilleur \(\lambda\) se fait par validation croisée.

\begin{Shaded}
\begin{Highlighting}[]
\CommentTok{\# tracer du GCV en fonction du lambda }
\FunctionTok{plot}\NormalTok{(lambda, mod\_ridge}\SpecialCharTok{$}\NormalTok{GCV, }\AttributeTok{pch =} \DecValTok{19}\NormalTok{, }\AttributeTok{col =} \DecValTok{6}\NormalTok{, }\AttributeTok{xlab =} \StringTok{"lambda"}\NormalTok{, }\AttributeTok{ylab =} \StringTok{"GCV"}\NormalTok{)}
\end{Highlighting}
\end{Shaded}

\includegraphics{BE3_partie1_incomplet_files/figure-latex/unnamed-chunk-11-1.pdf}

\begin{Shaded}
\begin{Highlighting}[]
\FunctionTok{plot}\NormalTok{(lambda[}\DecValTok{1}\SpecialCharTok{:}\DecValTok{40}\NormalTok{], mod\_ridge}\SpecialCharTok{$}\NormalTok{GCV[}\DecValTok{1}\SpecialCharTok{:}\DecValTok{40}\NormalTok{], }\AttributeTok{pch =} \DecValTok{19}\NormalTok{, }\AttributeTok{col =} \DecValTok{6}\NormalTok{, }\AttributeTok{xlab =} \StringTok{\textquotesingle{}lambda\textquotesingle{}}\NormalTok{, }\AttributeTok{ylab =} \StringTok{"GCV"}\NormalTok{)}
\end{Highlighting}
\end{Shaded}

\includegraphics{BE3_partie1_incomplet_files/figure-latex/unnamed-chunk-11-2.pdf}

\begin{Shaded}
\begin{Highlighting}[]
\CommentTok{\# choix du meilleur beta. On récupère la position (indice) où le GCV est minimum}
\NormalTok{indice }\OtherTok{=} \DecValTok{10}  \CommentTok{\# 0.1 : command which.min()}
\NormalTok{beta }\OtherTok{=}\NormalTok{ beta\_mat[, indice]}
\end{Highlighting}
\end{Shaded}

La remise à l'échelle est nécessaire pour la prédiction.

\begin{Shaded}
\begin{Highlighting}[]
\CommentTok{\#prédiction du modèle et remise à l\textquotesingle{}échelle pour comparer à diab.test}
\NormalTok{ypred }\OtherTok{=}  \FunctionTok{as.matrix}\NormalTok{((x\_extend}\SpecialCharTok{{-}}\FunctionTok{t}\NormalTok{(}\FunctionTok{matrix}\NormalTok{(}\FunctionTok{rep}\NormalTok{(mean.train[}\SpecialCharTok{{-}}\NormalTok{p2],}\DecValTok{300}\NormalTok{),p2}\DecValTok{{-}1}\NormalTok{,}\DecValTok{300}\NormalTok{)))}\SpecialCharTok{/}\FunctionTok{t}\NormalTok{(}\FunctionTok{matrix}\NormalTok{(}\FunctionTok{rep}\NormalTok{(sd.train[}\SpecialCharTok{{-}}\NormalTok{p2],}\DecValTok{300}\NormalTok{),p2}\DecValTok{{-}1}\NormalTok{,}\DecValTok{300}\NormalTok{)))}\SpecialCharTok{\%*\%}\FunctionTok{matrix}\NormalTok{(beta,p2}\DecValTok{{-}1}\NormalTok{,}\DecValTok{1}\NormalTok{)}\SpecialCharTok{*}\NormalTok{sd.train[p2] }\SpecialCharTok{+}\NormalTok{ mean.train[p2]}

\NormalTok{X\_test }\OtherTok{=} \FunctionTok{as.matrix}\NormalTok{(newx }\SpecialCharTok{{-}} \FunctionTok{t}\NormalTok{(}\FunctionTok{matrix}\NormalTok{(}\FunctionTok{rep}\NormalTok{(mean.train[}\SpecialCharTok{{-}}\NormalTok{p2], }\DecValTok{206}\NormalTok{), p2}\DecValTok{{-}1}\NormalTok{, }\DecValTok{206}\NormalTok{)))}
\NormalTok{X\_test }\OtherTok{=}\NormalTok{ X\_test }\SpecialCharTok{/} \FunctionTok{t}\NormalTok{(}\FunctionTok{matrix}\NormalTok{(}\FunctionTok{rep}\NormalTok{(sd.train[}\SpecialCharTok{{-}}\NormalTok{p2], }\DecValTok{206}\NormalTok{), p2}\DecValTok{{-}1}\NormalTok{, }\DecValTok{206}\NormalTok{))}
\NormalTok{Y\_pred\_ridge }\OtherTok{=}\NormalTok{ (}\FunctionTok{as.matrix}\NormalTok{(X\_test)}\SpecialCharTok{\%*\%}\FunctionTok{matrix}\NormalTok{(beta,p2}\DecValTok{{-}1}\NormalTok{,}\DecValTok{1}\NormalTok{))}\SpecialCharTok{*}\NormalTok{sd.train[p2]}\SpecialCharTok{+}\NormalTok{mean.train[p2]}

\NormalTok{RMSE\_ridge }\OtherTok{=} \FunctionTok{RMSE}\NormalTok{(housing.test[, p], Y\_pred\_ridge)}
\FunctionTok{c}\NormalTok{(RMSE\_lm\_inter, RMSE\_lasso, RMSE\_ridge)}
\end{Highlighting}
\end{Shaded}

\begin{verbatim}
## [1] 4.416989 4.581404 4.653380
\end{verbatim}

\hypertarget{moduxe8les-uxe0-base-darbres}{%
\section{Modèles à base d'arbres}\label{moduxe8les-uxe0-base-darbres}}

On commence par ajuster un modèle CART de la librairie Rpart.

\begin{Shaded}
\begin{Highlighting}[]
\CommentTok{\#{-}{-}{-}{-}{-}{-}{-}{-}{-}{-}{-}{-}{-}{-}{-}{-}{-}{-}{-}{-}{-}{-}{-}{-}{-}{-}{-}{-}{-}{-}{-}{-}{-}{-}{-}{-}{-}{-}{-}{-}{-}{-}}
\CommentTok{\#ajustement cart + random forest}
\CommentTok{\#{-}{-}{-}{-}{-}{-}{-}{-}{-}{-}{-}{-}{-}{-}{-}{-}{-}{-}{-}{-}{-}{-}{-}{-}{-}{-}{-}{-}{-}{-}{-}{-}{-}{-}{-}{-}{-}{-}{-}{-}{-}{-}}
\FunctionTok{library}\NormalTok{(rpart)}
\NormalTok{cont }\OtherTok{=} \FunctionTok{rpart.control}\NormalTok{(}\AttributeTok{cp =} \FloatTok{0.0001}\NormalTok{)  }\CommentTok{\# define minimum cp}
\NormalTok{mod\_tree }\OtherTok{\textless{}{-}} \FunctionTok{rpart}\NormalTok{(class }\SpecialCharTok{\textasciitilde{}}\NormalTok{., }\AttributeTok{data =}\NormalTok{ housing.train, }\AttributeTok{control =}\NormalTok{ cont)}
\FunctionTok{par}\NormalTok{(}\AttributeTok{mfrow =} \FunctionTok{c}\NormalTok{(}\DecValTok{1}\NormalTok{, }\DecValTok{1}\NormalTok{))}
\FunctionTok{plot}\NormalTok{(mod\_tree, }\AttributeTok{uniform =} \ConstantTok{TRUE}\NormalTok{, }\AttributeTok{margin =} \FloatTok{0.05}\NormalTok{)}
\FunctionTok{text}\NormalTok{(mod\_tree, }\AttributeTok{cex =} \FloatTok{0.6}\NormalTok{)}
\end{Highlighting}
\end{Shaded}

\includegraphics{BE3_partie1_incomplet_files/figure-latex/unnamed-chunk-13-1.pdf}

\begin{itemize}
\item
  A quoi correspond le paramètre \(\textit{cp}\) ? Il contrôle la
  complexité de l'arbre (complexity parameter).
\item
  Le choix de la valeur 0.0001 conduit-il à un arbre grossier ou
  détaillé ? Un \(\textit{cp}\) petit entraine un arbre profond avec
  beaucoup de noeuds.
\item
  Quelles sont les variables les plus influentes ? RM et LSTAT (look
  both the height and frequency)
\item
  Pourquoi n'est-il pas nécessaire ici de mettre en place un modèle avec
  interaction ? Trees already consider the interactions effects
\item
  Faire un choix du meilleur \(\textit{cp}\) par validation croisée
  (quatrième colonne de l'attribut \(\textit{cptable}\) d'un objet de
  type \(\textit{rpart}\)).
\end{itemize}

\begin{Shaded}
\begin{Highlighting}[]
\CommentTok{\#Apparent : apprentissage, X relative : par validation croisée}
\NormalTok{mod\_tree}\SpecialCharTok{$}\NormalTok{cptable }\CommentTok{\# ele testa cps maiores (árvores mais simples), antes de chegar em cp=0.0001}
\end{Highlighting}
\end{Shaded}

\begin{verbatim}
##              CP nsplit  rel error    xerror       xstd
## 1  0.5030656136      0 1.00000000 1.0114478 0.11451713
## 2  0.1662686916      1 0.49693439 0.5647737 0.05827073
## 3  0.0782727816      2 0.33066569 0.4107676 0.05333136
## 4  0.0414868184      3 0.25239291 0.2892056 0.03197709
## 5  0.0382249961      4 0.21090609 0.2802356 0.03224474
## 6  0.0148019951      5 0.17268110 0.2229881 0.03005734
## 7  0.0130550489      6 0.15787910 0.1998693 0.02349025
## 8  0.0111708195      7 0.14482405 0.1978706 0.02406715
## 9  0.0108478201      8 0.13365324 0.1894953 0.02378675
## 10 0.0051137140      9 0.12280542 0.1760483 0.02352917
## 11 0.0047095058     10 0.11769170 0.1752187 0.02347684
## 12 0.0027083741     11 0.11298220 0.1681780 0.02349925
## 13 0.0019607608     13 0.10756545 0.1697190 0.02364828
## 14 0.0015930841     14 0.10560469 0.1708474 0.02350104
## 15 0.0014471054     15 0.10401160 0.1718676 0.02353299
## 16 0.0013119307     16 0.10256450 0.1727270 0.02363719
## 17 0.0012952061     18 0.09994064 0.1710708 0.02362531
## 18 0.0009682042     19 0.09864543 0.1732953 0.02440385
## 19 0.0008134445     20 0.09767723 0.1753319 0.02442750
## 20 0.0005885364     21 0.09686378 0.1764818 0.02444670
## 21 0.0005644762     22 0.09627524 0.1764812 0.02444766
## 22 0.0004751045     23 0.09571077 0.1761164 0.02445440
## 23 0.0002569774     24 0.09523566 0.1766070 0.02447482
## 24 0.0001000000     25 0.09497869 0.1766223 0.02447450
\end{verbatim}

\begin{Shaded}
\begin{Highlighting}[]
\CommentTok{\#affichage uniquement cp\textgreater{}0.01}
\CommentTok{\#summary(mod\_tree, cp=0.01)}

\FunctionTok{par}\NormalTok{(}\AttributeTok{mfrow =} \FunctionTok{c}\NormalTok{(}\DecValTok{1}\NormalTok{, }\DecValTok{1}\NormalTok{)) }\CommentTok{\# one plot on one page}
\FunctionTok{rsq.rpart}\NormalTok{(mod\_tree) }\CommentTok{\# visualize cross{-}validation results}
\end{Highlighting}
\end{Shaded}

\begin{verbatim}
## 
## Regression tree:
## rpart(formula = class ~ ., data = housing.train, control = cont)
## 
## Variables actually used in tree construction:
## [1] AGE   CRIM  DIS   INDUS LSTAT NOX   RM    TAX  
## 
## Root node error: 22208/300 = 74.025
## 
## n= 300 
## 
##            CP nsplit rel error  xerror     xstd
## 1  0.50306561      0  1.000000 1.01145 0.114517
## 2  0.16626869      1  0.496934 0.56477 0.058271
## 3  0.07827278      2  0.330666 0.41077 0.053331
## 4  0.04148682      3  0.252393 0.28921 0.031977
## 5  0.03822500      4  0.210906 0.28024 0.032245
## 6  0.01480200      5  0.172681 0.22299 0.030057
## 7  0.01305505      6  0.157879 0.19987 0.023490
## 8  0.01117082      7  0.144824 0.19787 0.024067
## 9  0.01084782      8  0.133653 0.18950 0.023787
## 10 0.00511371      9  0.122805 0.17605 0.023529
## 11 0.00470951     10  0.117692 0.17522 0.023477
## 12 0.00270837     11  0.112982 0.16818 0.023499
## 13 0.00196076     13  0.107565 0.16972 0.023648
## 14 0.00159308     14  0.105605 0.17085 0.023501
## 15 0.00144711     15  0.104012 0.17187 0.023533
## 16 0.00131193     16  0.102564 0.17273 0.023637
## 17 0.00129521     18  0.099941 0.17107 0.023625
## 18 0.00096820     19  0.098645 0.17330 0.024404
## 19 0.00081344     20  0.097677 0.17533 0.024427
## 20 0.00058854     21  0.096864 0.17648 0.024447
## 21 0.00056448     22  0.096275 0.17648 0.024448
## 22 0.00047510     23  0.095711 0.17612 0.024454
## 23 0.00025698     24  0.095236 0.17661 0.024475
## 24 0.00010000     25  0.094979 0.17662 0.024474
\end{verbatim}

\includegraphics{BE3_partie1_incomplet_files/figure-latex/unnamed-chunk-14-1.pdf}
\includegraphics{BE3_partie1_incomplet_files/figure-latex/unnamed-chunk-14-2.pdf}

\begin{Shaded}
\begin{Highlighting}[]
\CommentTok{\# the best result is given by nsplit = 11}
\NormalTok{mod\_tree}\SpecialCharTok{$}\NormalTok{cptable[}\DecValTok{12}\NormalTok{, }\DecValTok{1}\NormalTok{]}
\end{Highlighting}
\end{Shaded}

\begin{verbatim}
## [1] 0.002708374
\end{verbatim}

\begin{itemize}
\tightlist
\item
  Elaguer l'arbre avec ce choix de \(\textit{cp}\) (fonction
  \(\textit{prune}\))
\end{itemize}

\begin{Shaded}
\begin{Highlighting}[]
\CommentTok{\#a completer :}
\NormalTok{mod\_tree\_pruned }\OtherTok{=} \FunctionTok{prune}\NormalTok{(mod\_tree, }\AttributeTok{cp =} \FloatTok{0.002708374}\NormalTok{)}
\FunctionTok{par}\NormalTok{(}\AttributeTok{mfrow =} \FunctionTok{c}\NormalTok{(}\DecValTok{1}\NormalTok{, }\DecValTok{1}\NormalTok{))}
\FunctionTok{plot}\NormalTok{(mod\_tree\_pruned)}
\FunctionTok{text}\NormalTok{(mod\_tree\_pruned, }\AttributeTok{use.n =} \ConstantTok{TRUE}\NormalTok{, }\AttributeTok{cex =} \FloatTok{0.7}\NormalTok{)}
\end{Highlighting}
\end{Shaded}

\includegraphics{BE3_partie1_incomplet_files/figure-latex/unnamed-chunk-15-1.pdf}

\begin{Shaded}
\begin{Highlighting}[]
\CommentTok{\#a completer :}
\NormalTok{y\_pred\_tree }\OtherTok{\textless{}{-}} \FunctionTok{predict}\NormalTok{(mod\_tree\_pruned, housing.test[, }\SpecialCharTok{{-}}\NormalTok{p])}
\CommentTok{\#a completer :}
\NormalTok{RMSE\_cart }\OtherTok{=} \FunctionTok{RMSE}\NormalTok{(housing.test[, p], y\_pred\_tree)}
\NormalTok{RMSE\_cart}
\end{Highlighting}
\end{Shaded}

\begin{verbatim}
## [1] 5.479321
\end{verbatim}

\begin{itemize}
\tightlist
\item
  Mettre en place un modèle de boosting en utilisant la routine \(gbm\)
  de la librairie \(gbm\).
\item
  Combien d'arbres doit-on considérer ? 726
\item
  Quelles sont les variables les plus influentes avec ce modèle ? RM et
  LSTAT
\end{itemize}

\begin{Shaded}
\begin{Highlighting}[]
\CommentTok{\#modèle boosted trees}
\FunctionTok{library}\NormalTok{(gbm)}
\end{Highlighting}
\end{Shaded}

\begin{verbatim}
## Loaded gbm 2.1.8.1
\end{verbatim}

\begin{Shaded}
\begin{Highlighting}[]
\NormalTok{mod\_boosted }\OtherTok{\textless{}{-}} \FunctionTok{gbm}\NormalTok{(class }\SpecialCharTok{\textasciitilde{}}\NormalTok{., }\AttributeTok{data =}\NormalTok{ housing.train, }\AttributeTok{n.trees =} \DecValTok{10000}\NormalTok{, }\AttributeTok{cv.folds =} \DecValTok{10}\NormalTok{)}
\end{Highlighting}
\end{Shaded}

\begin{verbatim}
## Distribution not specified, assuming gaussian ...
\end{verbatim}

\begin{Shaded}
\begin{Highlighting}[]
\FunctionTok{par}\NormalTok{(}\AttributeTok{mfrow =} \FunctionTok{c}\NormalTok{(}\DecValTok{1}\NormalTok{, }\DecValTok{1}\NormalTok{))}
\NormalTok{best.iter }\OtherTok{\textless{}{-}} \FunctionTok{gbm.perf}\NormalTok{(mod\_boosted, }\AttributeTok{method =} \StringTok{"cv"}\NormalTok{)}
\end{Highlighting}
\end{Shaded}

\includegraphics{BE3_partie1_incomplet_files/figure-latex/unnamed-chunk-17-1.pdf}

\begin{Shaded}
\begin{Highlighting}[]
\FunctionTok{par}\NormalTok{(}\AttributeTok{mfrow =} \FunctionTok{c}\NormalTok{(}\DecValTok{1}\NormalTok{, }\DecValTok{2}\NormalTok{))}
\FunctionTok{summary}\NormalTok{(mod\_boosted, }\AttributeTok{n.trees =} \DecValTok{1}\NormalTok{)          }\CommentTok{\# based on the first tree}
\end{Highlighting}
\end{Shaded}

\begin{verbatim}
##             var rel.inf
## LSTAT     LSTAT     100
## CRIM       CRIM       0
## ZN           ZN       0
## INDUS     INDUS       0
## CHAS       CHAS       0
## NOX         NOX       0
## RM           RM       0
## AGE         AGE       0
## DIS         DIS       0
## RAD         RAD       0
## TAX         TAX       0
## PTRATIO PTRATIO       0
\end{verbatim}

\begin{Shaded}
\begin{Highlighting}[]
\FunctionTok{summary}\NormalTok{(mod\_boosted, }\AttributeTok{n.trees =}\NormalTok{ best.iter)  }\CommentTok{\# based on the estimated best number of trees}
\end{Highlighting}
\end{Shaded}

\includegraphics{BE3_partie1_incomplet_files/figure-latex/unnamed-chunk-17-2.pdf}

\begin{verbatim}
##             var     rel.inf
## RM           RM 42.36820423
## LSTAT     LSTAT 32.15865779
## NOX         NOX  5.83833014
## CRIM       CRIM  4.46132510
## PTRATIO PTRATIO  4.10908279
## DIS         DIS  3.61888325
## TAX         TAX  2.30171792
## INDUS     INDUS  2.19312476
## AGE         AGE  1.88764093
## RAD         RAD  0.64305519
## CHAS       CHAS  0.38004279
## ZN           ZN  0.03993511
\end{verbatim}

\begin{Shaded}
\begin{Highlighting}[]
\FunctionTok{summary}\NormalTok{(mod\_boosted, }\AttributeTok{n.trees =} \DecValTok{3000}\NormalTok{)       }\CommentTok{\# based on something}
\end{Highlighting}
\end{Shaded}

\begin{verbatim}
##             var    rel.inf
## RM           RM 39.3864807
## LSTAT     LSTAT 26.8133470
## CRIM       CRIM  6.7066555
## NOX         NOX  5.7859931
## DIS         DIS  5.5650171
## PTRATIO PTRATIO  4.6317920
## AGE         AGE  3.7671004
## INDUS     INDUS  3.4174186
## TAX         TAX  2.7119535
## RAD         RAD  0.7809762
## CHAS       CHAS  0.3185178
## ZN           ZN  0.1147481
\end{verbatim}

\begin{Shaded}
\begin{Highlighting}[]
\CommentTok{\#print(pretty.gbm.tree(mod\_boosted,1))}
\CommentTok{\#print(pretty.gbm.tree(mod\_boosted,mod\_boosted$n.trees))}
\NormalTok{f.predict1 }\OtherTok{\textless{}{-}} \FunctionTok{predict}\NormalTok{(mod\_boosted, housing.test[, }\SpecialCharTok{{-}}\NormalTok{p], }\DecValTok{1}\NormalTok{)}
\NormalTok{f.predict2 }\OtherTok{\textless{}{-}} \FunctionTok{predict}\NormalTok{(mod\_boosted, housing.test[, }\SpecialCharTok{{-}}\NormalTok{p], }\DecValTok{3000}\NormalTok{)}
\NormalTok{f.predict3 }\OtherTok{\textless{}{-}} \FunctionTok{predict}\NormalTok{(mod\_boosted, housing.test[, }\SpecialCharTok{{-}}\NormalTok{p], best.iter)}
\NormalTok{RMSE\_b1 }\OtherTok{=}  \FunctionTok{RMSE}\NormalTok{(housing.test[, p], f.predict1)}
\NormalTok{RMSE\_b2 }\OtherTok{=}  \FunctionTok{RMSE}\NormalTok{(housing.test[, p], f.predict2)}
\NormalTok{RMSE\_b3 }\OtherTok{=}  \FunctionTok{RMSE}\NormalTok{(housing.test[, p], f.predict3)}
\FunctionTok{c}\NormalTok{(RMSE\_b1, RMSE\_b2, RMSE\_b3)}
\end{Highlighting}
\end{Shaded}

\begin{verbatim}
## [1] 9.618313 5.024867 4.853227
\end{verbatim}

\includegraphics{BE3_partie1_incomplet_files/figure-latex/unnamed-chunk-17-3.pdf}

On construit maintenant une procédure de bagging. L'idée est de moyenner
des arbres CART construits sur des échantillons bootstrap des données de
départ. Un échantillon Bootstrap est un tirage avec remise de N points
parmi les N points de l'échantillon initial.

\begin{Shaded}
\begin{Highlighting}[]
\CommentTok{\#modèle arbre bagging}
\NormalTok{cont }\OtherTok{=} \FunctionTok{rpart.control}\NormalTok{(}\AttributeTok{minsplit =} \DecValTok{2}\NormalTok{, }\AttributeTok{cp =} \FloatTok{0.0001}\NormalTok{)}
\NormalTok{B }\OtherTok{=} \DecValTok{2000}
\NormalTok{Y }\OtherTok{\textless{}{-}} \FunctionTok{matrix}\NormalTok{(}\DecValTok{0}\NormalTok{, }\FunctionTok{nrow}\NormalTok{(housing.test), B)}

\ControlFlowTok{for}\NormalTok{ (i }\ControlFlowTok{in} \DecValTok{1}\SpecialCharTok{:}\NormalTok{B) \{}
\NormalTok{  u }\OtherTok{=} \FunctionTok{sample}\NormalTok{(}\DecValTok{1}\SpecialCharTok{:}\FunctionTok{nrow}\NormalTok{(housing.train), }\FunctionTok{nrow}\NormalTok{(housing.train), }\AttributeTok{replace =} \ConstantTok{TRUE}\NormalTok{)}
\NormalTok{  appren }\OtherTok{\textless{}{-}}\NormalTok{ housing.train[u, ]}
\NormalTok{  mod\_tree }\OtherTok{\textless{}{-}} \FunctionTok{rpart}\NormalTok{(class }\SpecialCharTok{\textasciitilde{}}\NormalTok{., }\AttributeTok{data =}\NormalTok{ appren, }\AttributeTok{control =}\NormalTok{ cont)}
\NormalTok{  Y[, i] }\OtherTok{\textless{}{-}} \FunctionTok{predict}\NormalTok{(mod\_tree, }\AttributeTok{newdata =}\NormalTok{ housing.test)}
\NormalTok{\}}

\CommentTok{\#Y est de taille 206*B, 206 etant le nombre de points tests}
\CommentTok{\#a completer :}
\NormalTok{Y\_pred\_bag }\OtherTok{=} \FunctionTok{apply}\NormalTok{(Y, }\DecValTok{1}\NormalTok{, mean)  }\CommentTok{\# mean of each tree\textasciigrave{}s result}
\CommentTok{\#a completer :}
\NormalTok{RMSE\_bag }\OtherTok{=} \FunctionTok{RMSE}\NormalTok{(housing.test[, p], Y\_pred\_bag)}
\NormalTok{RMSE\_bag}
\end{Highlighting}
\end{Shaded}

\begin{verbatim}
## [1] 4.864175
\end{verbatim}

Construire un modèle de Type Random Forest. - quelle est la différence
entre un modèle type bagging et un modèle de forets aléatoires ?

\begin{Shaded}
\begin{Highlighting}[]
\CommentTok{\#modèle arbres random forest}
\FunctionTok{library}\NormalTok{(randomForest)}
\end{Highlighting}
\end{Shaded}

\begin{verbatim}
## randomForest 4.7-1.1
\end{verbatim}

\begin{verbatim}
## Type rfNews() to see new features/changes/bug fixes.
\end{verbatim}

\begin{verbatim}
## 
## Attaching package: 'randomForest'
\end{verbatim}

\begin{verbatim}
## The following object is masked from 'package:ggplot2':
## 
##     margin
\end{verbatim}

\begin{Shaded}
\begin{Highlighting}[]
\NormalTok{mod\_RF }\OtherTok{\textless{}{-}} \FunctionTok{randomForest}\NormalTok{(class}\SpecialCharTok{\textasciitilde{}}\NormalTok{., }\AttributeTok{data =}\NormalTok{ housing.train, }\AttributeTok{ntree =} \DecValTok{300}\NormalTok{, }\AttributeTok{sampsize =} \FunctionTok{nrow}\NormalTok{(housing.train))}

\FunctionTok{summary}\NormalTok{(mod\_RF)}
\end{Highlighting}
\end{Shaded}

\begin{verbatim}
##                 Length Class  Mode     
## call              5    -none- call     
## type              1    -none- character
## predicted       300    -none- numeric  
## mse             300    -none- numeric  
## rsq             300    -none- numeric  
## oob.times       300    -none- numeric  
## importance       12    -none- numeric  
## importanceSD      0    -none- NULL     
## localImportance   0    -none- NULL     
## proximity         0    -none- NULL     
## ntree             1    -none- numeric  
## mtry              1    -none- numeric  
## forest           11    -none- list     
## coefs             0    -none- NULL     
## y               300    -none- numeric  
## test              0    -none- NULL     
## inbag             0    -none- NULL     
## terms             3    terms  call
\end{verbatim}

\begin{Shaded}
\begin{Highlighting}[]
\FunctionTok{importance}\NormalTok{(mod\_RF)}
\end{Highlighting}
\end{Shaded}

\begin{verbatim}
##         IncNodePurity
## CRIM       1289.00260
## ZN          194.06135
## INDUS      1409.38286
## CHAS         41.82852
## NOX        1309.59676
## RM         7471.05953
## AGE         656.74447
## DIS        1036.39422
## RAD         144.51651
## TAX        1059.09902
## PTRATIO    1306.87398
## LSTAT      5852.45326
\end{verbatim}

\begin{Shaded}
\begin{Highlighting}[]
\FunctionTok{plot}\NormalTok{(mod\_RF)}
\end{Highlighting}
\end{Shaded}

\includegraphics{BE3_partie1_incomplet_files/figure-latex/unnamed-chunk-19-1.pdf}

\begin{Shaded}
\begin{Highlighting}[]
\FunctionTok{which}\NormalTok{(mod\_RF}\SpecialCharTok{$}\NormalTok{trees }\SpecialCharTok{==} \FunctionTok{min}\NormalTok{(mod\_RF}\SpecialCharTok{$}\NormalTok{trees), }\AttributeTok{arr.ind =} \ConstantTok{TRUE}\NormalTok{)}
\end{Highlighting}
\end{Shaded}

\begin{verbatim}
## Warning in min(mod_RF$trees): no non-missing arguments to min; returning Inf
\end{verbatim}

\begin{verbatim}
## integer(0)
\end{verbatim}

\begin{Shaded}
\begin{Highlighting}[]
\CommentTok{\#a completer :}
\NormalTok{Y\_pred\_RF }\OtherTok{\textless{}{-}} \FunctionTok{predict}\NormalTok{(mod\_RF,housing.test)}
\CommentTok{\#a completer :}
\NormalTok{RMSE\_RF }\OtherTok{=} \FunctionTok{RMSE}\NormalTok{(housing.test[, p], Y\_pred\_RF)}
\NormalTok{RMSE\_RF}
\end{Highlighting}
\end{Shaded}

\begin{verbatim}
## [1] 4.931825
\end{verbatim}

\hypertarget{muxe9thode-de-ruxe9duction-de-dimension-par-orthogonalisation}{%
\section{Méthode de réduction de dimension par
orthogonalisation}\label{muxe9thode-de-ruxe9duction-de-dimension-par-orthogonalisation}}

On s'intéresse maintenant aux modèles PCR et PLS construits à partir de
la matrice augmentée des interactions.

\begin{Shaded}
\begin{Highlighting}[]
\FunctionTok{library}\NormalTok{(pls)}
\end{Highlighting}
\end{Shaded}

\begin{verbatim}
## 
## Attaching package: 'pls'
\end{verbatim}

\begin{verbatim}
## The following object is masked from 'package:caret':
## 
##     R2
\end{verbatim}

\begin{verbatim}
## The following object is masked from 'package:DiceEval':
## 
##     R2
\end{verbatim}

\begin{verbatim}
## The following object is masked from 'package:stats':
## 
##     loadings
\end{verbatim}

\begin{Shaded}
\begin{Highlighting}[]
\CommentTok{\#PCR}
\NormalTok{housing.pcr }\OtherTok{\textless{}{-}} \FunctionTok{pcr}\NormalTok{(class }\SpecialCharTok{\textasciitilde{}}\NormalTok{ .}\SpecialCharTok{\^{}}\DecValTok{2}\NormalTok{, }\DecValTok{60}\NormalTok{, }\AttributeTok{data =}\NormalTok{ housing.train)}
\end{Highlighting}
\end{Shaded}

Faire le choix du nombre de composantes par validation croisée en
utilisant la routine crossval.

\begin{Shaded}
\begin{Highlighting}[]
\CommentTok{\#a completer :}
\NormalTok{housing.cv }\OtherTok{\textless{}{-}} \FunctionTok{crossval}\NormalTok{(housing.pcr,}\AttributeTok{segments =} \DecValTok{10}\NormalTok{)}
\FunctionTok{plot}\NormalTok{(}\FunctionTok{MSEP}\NormalTok{(housing.cv))}
\end{Highlighting}
\end{Shaded}

\includegraphics{BE3_partie1_incomplet_files/figure-latex/unnamed-chunk-21-1.pdf}

Faire la prédiction du modèle pour le nombre de composantes
sélectionnées.

\begin{Shaded}
\begin{Highlighting}[]
\CommentTok{\#a completer :}
\NormalTok{nbcomp }\OtherTok{=} \DecValTok{36}
\NormalTok{Y\_PCR }\OtherTok{=} \FunctionTok{predict}\NormalTok{(housing.pcr, }\AttributeTok{newdata =}\NormalTok{ housing.test, }\AttributeTok{ncomp =}\NormalTok{ nbcomp, }\AttributeTok{type =} \StringTok{"response"}\NormalTok{)}
\CommentTok{\#a completer :}
\NormalTok{RMSE\_PCR }\OtherTok{=} \FunctionTok{RMSE}\NormalTok{(housing.test[, p], Y\_PCR)}
\NormalTok{RMSE\_PCR}
\end{Highlighting}
\end{Shaded}

\begin{verbatim}
## [1] 5.487672
\end{verbatim}

Tracer les premières fonctions propres et interpréter.

\begin{Shaded}
\begin{Highlighting}[]
\NormalTok{housing.pcr }\OtherTok{\textless{}{-}} \FunctionTok{pcr}\NormalTok{(class }\SpecialCharTok{\textasciitilde{}}\NormalTok{ .}\SpecialCharTok{\^{}}\DecValTok{2}\NormalTok{, nbcomp, }\AttributeTok{data =}\NormalTok{ housing.train)}
\CommentTok{\#coef(housing.pcr)}

\CommentTok{\#tracer des premières fonctions propres }
\FunctionTok{par}\NormalTok{(}\AttributeTok{mfrow =}\FunctionTok{c}\NormalTok{(}\DecValTok{1}\NormalTok{,}\DecValTok{1}\NormalTok{))}
\FunctionTok{plot}\NormalTok{(}\DecValTok{1}\SpecialCharTok{:}\NormalTok{(p}\DecValTok{{-}1}\NormalTok{),housing.pcr}\SpecialCharTok{$}\NormalTok{loadings[}\DecValTok{1}\SpecialCharTok{:}\NormalTok{(p}\DecValTok{{-}1}\NormalTok{),}\DecValTok{1}\NormalTok{],}\AttributeTok{type =}\StringTok{"l"}\NormalTok{,}\AttributeTok{col =} \DecValTok{1}\NormalTok{,}\AttributeTok{ylim =}\FunctionTok{c}\NormalTok{(}\SpecialCharTok{{-}}\FloatTok{0.01}\NormalTok{,}\FloatTok{0.01}\NormalTok{))}
\FunctionTok{points}\NormalTok{(}\DecValTok{1}\SpecialCharTok{:}\NormalTok{(p}\DecValTok{{-}1}\NormalTok{),housing.pcr}\SpecialCharTok{$}\NormalTok{loadings[}\DecValTok{1}\SpecialCharTok{:}\NormalTok{(p}\DecValTok{{-}1}\NormalTok{),}\DecValTok{1}\NormalTok{],}\AttributeTok{col =} \DecValTok{1}\NormalTok{)}
\ControlFlowTok{for}\NormalTok{ (i }\ControlFlowTok{in} \DecValTok{2}\SpecialCharTok{:}\DecValTok{4}\NormalTok{) }
\NormalTok{\{}
  \FunctionTok{points}\NormalTok{(}\DecValTok{1}\SpecialCharTok{:}\NormalTok{(p}\DecValTok{{-}1}\NormalTok{),housing.pcr}\SpecialCharTok{$}\NormalTok{loadings[}\DecValTok{1}\SpecialCharTok{:}\NormalTok{(p}\DecValTok{{-}1}\NormalTok{),i],}\AttributeTok{col =}\NormalTok{ i)}
  \FunctionTok{lines}\NormalTok{(}\DecValTok{1}\SpecialCharTok{:}\NormalTok{(p}\DecValTok{{-}1}\NormalTok{),housing.pcr}\SpecialCharTok{$}\NormalTok{loadings[}\DecValTok{1}\SpecialCharTok{:}\NormalTok{(p}\DecValTok{{-}1}\NormalTok{),i],}\AttributeTok{col =}\NormalTok{ i)}
\NormalTok{\}}
\end{Highlighting}
\end{Shaded}

\includegraphics{BE3_partie1_incomplet_files/figure-latex/unnamed-chunk-23-1.pdf}

\begin{Shaded}
\begin{Highlighting}[]
\CommentTok{\#PLS}
\NormalTok{housing.pls }\OtherTok{\textless{}{-}} \FunctionTok{plsr}\NormalTok{(class }\SpecialCharTok{\textasciitilde{}}\NormalTok{ .}\SpecialCharTok{\^{}}\DecValTok{2}\NormalTok{, }\DecValTok{60}\NormalTok{, }\AttributeTok{data =}\NormalTok{ housing.train)}
\NormalTok{housing.pls.cv }\OtherTok{\textless{}{-}} \FunctionTok{crossval}\NormalTok{(housing.pls,}\AttributeTok{segments =}\DecValTok{10}\NormalTok{)}
\FunctionTok{plot}\NormalTok{(}\FunctionTok{MSEP}\NormalTok{(housing.pls.cv))}
\end{Highlighting}
\end{Shaded}

\includegraphics{BE3_partie1_incomplet_files/figure-latex/unnamed-chunk-24-1.pdf}

\begin{Shaded}
\begin{Highlighting}[]
\CommentTok{\#a completer :}
\NormalTok{Y\_PLS }\OtherTok{=} \FunctionTok{predict}\NormalTok{(housing.pls, }\AttributeTok{newdata =}\NormalTok{ housing.test, }\AttributeTok{ncomp =}\NormalTok{ nbcomp, }\AttributeTok{type =} \StringTok{"response"}\NormalTok{)}
\CommentTok{\#a completer :}
\NormalTok{RMSE\_PLS }\OtherTok{=} \FunctionTok{RMSE}\NormalTok{(housing.test[, p], Y\_PLS)}
\NormalTok{RMSE\_PLS}
\end{Highlighting}
\end{Shaded}

\begin{verbatim}
## [1] 4.993442
\end{verbatim}

\end{document}
